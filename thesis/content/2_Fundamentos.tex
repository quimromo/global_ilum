\chapter{Fundamentos teóricos}

\section{Unidades Radiométricas}
Se conoce como radiometría al estudio de las radiaciones electromagneticas. Ya que la luz visible es una onda electromagnetica los algoritmos de renderizado que buscan el realismo se fundamentan sobre conceptos radiométricos. Por ello en esta sección haremos una pequeña introducción sobre algunos conceptos básicos que nos permitiran entender mejor los algoritmos de iluminación global.  
\subsection{Flujo}

El flujo radiometrico mide la cantidad de energia radiante por unidad de tiempo. Sus unidades son Watts o Joules/segundo.

\begin{equation}
\Phi = \frac{dQ(t)}{dt}
\end{equation}

\subsection{Irradiancia}
La irradiancia representa el flujo incidente en una superficie y se mide como el flujo radiante por unidad de area y sus unidades son de $W/m^2$ 

\begin{equation}
E = \frac{d\Phi}{dA}
\end{equation}

\clearpage

\subsection{Angulo solido}
El angulo solido no es una unidad radiométrica en si mismo pero es un concepto geométrico necesario para poder explicar otros conceptos radiométricos además de otros apartados del presente trabajo.

Podemos entender el concepto de angulo solido como la extension del angulo a las tres dimensiones.
\begin{figure}[h]
\centering
\includegraphics[width=2in]{Solid_Angle.png}
\caption{Definición de angulo solido \cite{Haade}}
\end{figure}

El angulo solido se mide como el área proyectada sobre una esfera de radio unitario. Sus unidades son adimensionales y son llamadas stereorradianes $[sr]$.
\begin{equation}
\Omega = \frac{A}{r^2}
\end{equation}


Usando coordenadas esféricas $\Theta = (\phi , \theta )$ podemos definir el angulo solido diferencial como:

\begin{equation}
d \omega _ \Theta = \sin \theta d \theta d \phi
\end{equation}

Informalmente resulta sencillo entender el angulo solido si pensamos en \emph{cuan grande se ve un objeto}. Supongamos una superficie perpendicular a la dirección de visión del observador: si este objeto esta muy cerca, diremos que subtiende un angulo solido mayor que la misma superficie a una mayor distancia en la misma dirección.

\clearpage

\subsection{Radiancia}

La radiancia, también llamada intensidad por algunos autores \cite{Kajiya1986, Immel1986}, es probablemente la unidad radiometrica mas importante en lo que concierne al presente trabajo y a muchos de los algoritmos de iluminación global ya que su valor es invariante a lo largo de la longitud de un rayo.

Esta unidad mide la irradiancia por unidad de angulo solido.

\begin{equation}
L = \frac{dE}{d\omega} = \frac{d^2\Phi}{d\omega dA\cos \theta} 
\end{equation}


\clearpage

\section{BRDF}

La función de distribución de reflectancia bidireccional (de ahora en adelante BRDF, por sus siglas en inglés), definida por primera vez por \cite{Nicodemus1965} Nicodemus (1965), es un función que define la respuesta a la luz de una superficie opaca, tomando como parámetros dos vectores unitarios que definen las direcciones de entrada y salida de la luz. Más formalmente, la BRDF mide la relación entre la radiancia diferencial reflejada en la dirección de salida y la irradiancia diferencial entrante en el ángulo sólido diferencial alrededor del vector de entrada

\begin{equation}
f(x, l, v)=\frac{dL(x \to v)}{dE(x \gets l)} = \frac{dL(x \to v)}{L(x \gets l) \cos\theta d\omega_i} 
\end{equation}

donde $l$ es el vector unitario que apunta en la dirección opuesta a la de entrada de la luz y $v$ es el vector unitario que apunta en la dirección de salida de la luz.

La BRDF solo esta definida para vectores $l$ y $v$ tales que $n \cdot v > 0, n \cdot l > 0$, siendo $n$ la normal de la superficie.

\begin{figure}[h]
\centering
\includegraphics[scale=0.5]{Rendering_eq.png}
\caption{BRDF $l = \omega_i, v = \omega_o$ }
\end{figure}


Para obtener la radiancia total reflejada en un punto $x$ en la dirección saliente $v$ es necesario integrar sobre el angulo solido en el dominio de la hemiesfera centrada en $x$.

\begin{equation}
\label{eq:radiance_integral}
L _ o = \int_{\Omega_x} f(x, l, v) L_i(l) (l \cdot n) d\omega_i 
\end{equation}

\clearpage

\subsection{Propiedades de la BRDF}

Una BRDF debe cumplir ciertas propiedades para que sea físicamente plausible.
En primer lugar debe cumplir la ley de conservación de la energía. En el caso que nos ocupa esto significa que una superficie puede absorber luz, transformándola en calor, o puede reflejarla pero en ningún caso puede reflejar mas energía lumínica que la que recibe.

\begin{equation}
\forall l, \int_{\Omega_x} f(x,l,v) (n \cdot v) d\omega_o \leq 1
\end{equation}

En términos informales esta ecuación significa que la integral de toda la luz reflejada debido a un rayo de luz entrante nunca podrá ser superior a la luz entrante por ese rayo.

\medskip
Además también debe cumplir el \emph{principio de reciprocidad de Helmholtz}, esto significa que si intercambiamos los vectores $l$ y $v$ su valor se mantiene. Este hecho cobra sentido si pensamos que la BRDF es una característica intrínseca de cada material y que al intercambiar los vectores $v$ y $l$ el angulo entre ellos sigue siendo el mismo.

\begin{equation}
f(x, l, v) = f(x, v, l)
\end{equation} 

\subsection{Isotropía y anisotropía de la BRDF}

Una BRDF isotropica es aquella en la que su valor se mantendrá constante si aplicamos la misma rotación a $v$ y a $l$ alrededor de la normal de la superficie. Por el contrario una BRDF anisotropia cambiara su valor dependido de la rotación de $v$ y $l$ alrededor de la normal.

\clearpage

\subsection{Modelo de Phong modificado}
Una de las BRDFs mas usadas en los algoritmos de ray tracing estocástico es la BRDF basada en el modelo de Phong modificado, que fue descrito por primera vez por Lewis (1994) \nocite{Lewis1994} y posteriormente explorado en mas profundidad en \cite{Lafortune1994}. Esta BRDF esta basado en el conocido modelo de reflexión local de Phong \cite{Phong1975} que fue adaptado por Lewis para cumplir con el principio de conservación de la energía.

En la forma usada por Lafortune y Willems la función aparece como:

\begin{equation}
f(x, l, v) = \frac{k_d}{\pi} + k_s \frac{n + 2}{2 \pi} \cos^n\alpha
\end{equation}

donde $\alpha$ es el angulo entre la dirección de reflexión especular perfecta y $v$.
En esta forma la función esta normalizada para conservar la energía pero además para que esto sea cierto se debe cumplir $k_d + k_s \leq 1$.

 

\subsection{Modelo de Blinn-Phong}

\subsection{Modelo de Cook-Torrance}

\subsection{Modelo de Ward}

\subsection{Modelo de Beckmann}

\clearpage

\section{Ecuación de renderizado}

La ecuación de renderizado fue desarrollada en los años 80 simultaneamente y de forma independiente por distintos autores \cite{Kajiya1986, Immel1986}. Se trata de una ecuación integral que unifica y formaliza los distintos algoritmos de renderizado, ya que hasta ese momento no existía un marco de trabajo teórico común.

\medskip
Existen varias versiones de esta ecuación, según el autor que la use, que en general se pueden clasificar en dos tipos: las que integran sobre la hemiesfera, que se corresponde con la ecuación propuesta por Immel y las que integran sobre la unión de las superficies de la escena, que es la version propuesta por Kajiya.
\medskip

Consideremos la ecuación \ref{eq:radiance_integral}  y consideremos que además de dispersar luz una superficie también puede emitir luz, siendo $L_e$ la radiancia de la luz emitida, entonces tenemos la ecuación de renderizado.

\begin{equation}
L _ o = L_e + \int_{\Omega_x} f(x, l, v) L_i(l) (l \cdot n) d\omega_i 
\end{equation}

Es decir, que la radiancia total $L_o$ que sale de un punto $x$ es igual a la suma de la radiancia emitida por ese punto en la dirección de salida $v$ mas la integral de toda la radiancia que llega a ese punto y es reflejada en la dirección de salida.
\medskip

Lo significativo de esta ecuación es que resulta muy intuitivo derivar algoritmos de renderizado de la misma: se evalúa para cada punto a pintar y se evalúa $L_i$ recursivamente hasta que se cumpla determinada condición.
\medskip

El problema es que no parece factible encontrar una solución analítica de esta ecuación y por este motivo se aplican métodos de integración numérica para aproximar una solución.

\clearpage

\section{El método de montecarlo}

El método de montecarlo se trata de un método de integración numérico para integrales definidas sobre un dominio de dimension arbitraria, del tipo:
\begin{equation}
I = \int_D f(x)dx , D \subseteq \mathbb{R}^m
\end{equation}

Sabemos que la esperanza de una función continua se define como la integral de la función por la probabilidad de $x$. Y que podemos estimar la esperanza calculando la media de los valores que toma la función en puntos aleatorios escogidos independientemente y con la misma distribución.

\begin{equation}
E(f(x)) = \int f(x)p(x)dx \approx \frac{1}{N} \sum _{i=1} ^N f(x_i) 
\end{equation}


El método de montecarlo se basa en este hecho para estimar el valor de una integral definida tomando muestras aleatorias sobre el dominio $x_1, x_2, ..., x_n \in D$ y aplicando:

\begin{equation}
\label{eq:montecarlo}
I = \int_D f(x)dx \approx \frac{1}{N} \sum _{i=1} ^N \frac{f(x_i)}{p(x_i)} 
\end{equation}

Siendo $p(x_i)$ la probabilidad de tomar una muestra $x_i$ concreta de entre todas las posibles en el dominio $D$. En el caso de tomar las muestras sobre una distribución uniforme en $D$:

\begin{equation}
\forall x_i, p(x_i) = \frac{1}{\int _D dx}
\end{equation}

\begin{equation}
I \approx \frac{\int _D dx}{N} \sum_{i=1} ^N f(x_i)
\end{equation}

El error en una estimación de este tipo se reduce a medida que $N$ crece.
\clearpage

\subsection{Muestreo de importancia}
    
Otra forma de reducir el error a parte de tomar mas muestras es tomarlas de forma mas inteligente. Anteriormente hemos supuesto que tomamos las muestras de una distribución uniforme sobre el dominio pero el método de montecarlo no impone ninguna limitación en este aspecto. Lo que implica que podemos tomar las muestras de otro tipo de distribuciones que sean mas apropiadas para cada caso. Por ejemplo tomando mas muestras en aquellas partes del dominio de integración que sean mas interesantes o importantes para nuestros propósitos.

\medskip

Para ello basta con tomar las muestras $x_1, x_2, ..., x_n$ según la distribución usada y substituir $p(x_i)$ en la ecuación \ref{eq:montecarlo} por la probabilidad correspondiente.

\subsection{Muestreo estratificado}

El muestreo estratificado es otro método para reducir la variancia de la estimación. En este caso lo que se hace es dividir el dominio de la integral en regiones y aplicar montecarlo para cada región. 

\clearpage

\section{Aplicaciones del muestreo de importancia}

En esta sección veremos dos aplicaciones practicas en los algoritmos de ray tracing estocástico del muestreo de importancia y como usando esta técnica es posible reducir el numero de muestras necesarias para obtener una buena estimación de la integral.


\subsection{Muestreo de la BRDF}

Aunque algunos de los conceptos explicados aquí son aplicables a otros modelos, para esta explicación nos centraremos en la BRDF de Phong modificada, ya que es la usada en nuestra implementación además de una de las mas exploradas \cite{Lafortune1994}.

\medskip

El problema que nos ocupa se nos presenta cuando tenemos que estimar la luz que llega a un punto del espacio. Una aproximación \emph{naive}, sin usar muestreo de importancia, seria muestrear uniformemente direcciones en la hemiesfera y evaluar la BRDF para cada dirección. Esto puede funcionar bien cuando se trata de materiales puramente difusos, sin componente especular o con un lóbulo especular muy abierto. Por el contrario en el caso de encontrarnos con una superficie altamente especular, como por ejemplo un metal, la cantidad de muestras necesarias para obtener una estimación decente seria desorbitada ya que la probabilidad de elegir una dirección dentro del lóbulo especular seria muy pequeña.

\medskip
Cuando nos encontremos en la tesitura de tener que muestrear proporcionalmente a la BRDF lo primero que habrá que tener en cuenta sera decidir si muestrear la parte difusa o la parte especular de la BRDF de forma proporcional a las características del material en cuestión. Es decir que para un rayo, la probabilidad de muestrear el lóbulo difuso sera de $k_d$, la probabilidad de muestrear el lóbulo especular sera $k_s$ y la probabilidad de ser absorbido sera $1 - k_d - k_s$.

\medskip

Para ello generaremos un numero aleatorio $0 \leq r \leq 1$ y si
$r \leq k_s$ muestrearemos la parte especular,
si $k_s < r \leq k_s + k_d$ muestrearemos la parte difusa
y si $k_s + k_d < r$ el rayo sera absorbido por el material.

\medskip

Una vez decidido el evento a evaluar procederemos de forma distinta según cada caso. En el caso de evaluar la parte difusa podemos muestrear uniformente sobre la hemiesfera o podemos muestrear usando una distribución coseno como defienden algunos autores.

\clearpage

Una forma sencilla de obtener puntos distribuidos con densidad de coseno en la hemiesfera es generar puntos uniformemente sobre un circulo unitario y luego proyectarlos a la hemiesfera.
Para ello generamos dos números aleatorios $r_1$ y $r_2$ en el intervalo $[0, 1]$ y hacemos:

\smallskip

$x = r_1 \cos (2\pi r_2)$

\smallskip

$y = r_1 \sin (2\pi r_2)$

\smallskip

$z = \sqrt{1 - x^2 - y^2}$

\smallskip

$l = (x, y, z)$

\medskip

En este caso la funcion de densidad de probabilidad (pdf) es

\smallskip

\begin{equation}
pdf(l_{diff}) = \frac{(l \cdot n)}{\pi}
\end{equation}

La parte mas interesante es muestrear el lóbulo especular de la BRDF. Una buena explicación del procedimiento se encuentra en \cite{Lafortune1994}. En este caso la función de densidad de probabilidad es

\begin{equation}
pdf(l_{spec}) = \frac{n + 1}{2\pi}\cos^n\theta
\end{equation}

\begin{equation}
(\phi, \theta) = (2\pi r_1, \arccos(r_2^{\frac{1}{n+1}}) 
\end{equation}

\clearpage

\subsection{Muestreo del angulo solido subtendido}

Como en este trabajo estamos calculando la integral de la luz dispersada integrando la hemiesfera, cuando tengamos que muestrear un fuente de luz sera necesario integrar el angulo solido subtendido por dicha fuente de luz. A priori, no sabemos si la fuente de luz estara ocluida pero si que podemos saber que angulo subtiende con respecto al punto sobre el que estamos calculando la luz que llega, por lo tanto la idea sera lanzar solo rayos dentro de este angulo solido para que vayan dirigidos hacia la fuente de luz.

\medskip

Para proceder de este modo sera necesario calcular el angulo solido y definir una estrategia de muestreo para cada tipo de luz presente. En este trabajo hemos usado fuentes de luz en forma de cúpula celestial, esféricas, paralelogramos y triángulos.

\subsubsection{Cúpula}

\subsubsection{Esfera}

\subsubsection{Paralelogramo}

\subsubsection{Triangulo}




