\chapter{Introducción}

\section{Contexto}

En el entorno de la imagen generada por computador siempre ha sido un reto tratar de generar imágenes lo más realistas posibles. Para ello un gran número de investigadores se han dedicado a diseñar algoritmos que simulan o imiten el comportamiento y la interacción de la luz con los materiales. Estos algoritmos que tratan de simular de forma realista el comportamiento de la luz son generalmente conocidos como algoritmos de iluminación global.

Estos algoritmos, por lo general, suelen tener una complejidad computacional muy elevada y el tiempo de cómputo necesario para obtener un resultado satisfactorio en escenas complejas era un factor limitador en su aplicación práctica. Por ello las aplicaciones que hacen uso de gráficos 3D en tiempo real típicamente se centran en la iluminación local o directa de los objetos de la escena y simulan la iluminación indirecta mediante técnicas que aun sin tener un fundamento físico ofrecen una mayor credibilidad para el ojo humano. Estas técnicas suelen ser algoritmos de postprocesado que se aplican en espacio de pantalla, por ejemplo “ambient occlusion” o “directional occlusion”. 

Sin embargo en los últimos años se han realizado grandes avances en las arquitecturas de las unidades de procesamiento de gráficos (GPUs), en especial la gran capacidad de cómputo en paralelo debido al elevado número de microprocesadores que forman estos dispositivos. Con tal de aprovechar estos avances en el hardware, los fabricantes de GPU han desarrollado librerías de computo generico (OpenCL, CUDA) que ofrecen gran libertad al programador para implementar sus propios algoritmos.

Estas mejoras han permitido realizar implementaciones de algoritmos de iluminación global en las GPUs que son mucho más rápidos que las implementaciones típicas en la CPU permitiendo reducir el tiempo de cómputo de varias horas o días a minutos e incluso a ratios interactivos dependiendo de la GPU y algoritmos utilizados.

\clearpage

\section{Algoritmos de iluminación global}

Se conoce como algoritmos de iluminación global aquellos que tratan de simular distintos aspectos del comportamiento de la luz en su interacción con los objetos de una escena tridimensional. Algunos de ellos están pensados y optimizados para fenómenos concretos mientras que otros tratan de recrear fielmente todos los aspectos del transporte de luz.

En esta sección revisaremos por encima algunos de los algoritmos clásicos.

 
\subsection{Radiosity (Goral et al. 1984)}

Radiosity fue el primero de los algoritmos de iluminación global que se desarrollaron. Inicialmente el algoritmo fue desarrollado en los años 1950 para aplicarlo al problema de la transferencia de calor. En 1984 fue modificado y adaptado por Cindy M. Goral, Kenneth E. Torrance, Donald P. Greenberg y Bennett Battaile, investigadores de la universidad de Cornell para su aplicación en la generación de imagen sintética.

Este algoritmo trata de resolver el problema de la iluminación indirecta entre superficies puramente difusas o Lambertianas.
