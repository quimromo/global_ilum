\chapter{Conclusiones y resultados}


\section{Valoración de los resultados obtenidos}


El campo de estudio en el que se sitúa el presente trabajo es muy amplio y ha sido muy estudiado por un gran numero de investigadores. A medida que profundizábamos en la materia de estudio iban surgiendo cada ves mas detalles, mejoras y variaciones en la teoría que incrementaban el abasto del proyecto y generaban nuevas dudas. Por ello hemos llegado a un punto en el que, por el abasto de un TFG, hemos tenido que empezar a descartar cosas que al autor le hubiese gustado investigar en mas profundidad. Aun así, valoramos positivamente los resultados obtenidos pues se ha alcanzado el objetivo principal de este trabajo que era implementar un algoritmo de renderizado realista. Además la implementación en la GPU ha demostrado ser bastante rápida, mas aun si consideramos que la GPU usada en el desarrollo de este trabajo y en las pruebas realizadas es de una gama baja entre las presentes en el mercado.

\section{Posibles mejoras}

Planteamos tres grupos de posibles mejores bien diferenciados: En primer lugar tenemos las mejoras al algoritmo usado. El algoritmo de path tracing que hemos implementado es una de sus versiones mas basicas y tal como hemos comentado en la introducción existen variaciones del mismo que ofrecen mejores resultados en un menor numero de muestras. Por ejemplo, una primera mejora seria implementar el transporte de luz bidireccional que ofrecería mejores resultados cuando se trata de iluminar escenas en las que la fuente de luz esta muy escondida con respecto a la zona enfocada por la cámara. Otra importante mejora seria explorar e implementar modelos de BRDFs mas realistas, pues el que hemos usado es de los mas básicos que hay. Además, en este trabajo hemos omitido el importante fenómeno de la refractancia de la luz.

\medskip

Otro grupo de mejoras, mas allá de este algoritmo en concreto, seria buscar formas de combinar distintos algoritmos. Aunque en principio el algoritmo utilizado es capaz de simular los fenómenos de la luz mas habituales esto no significa que sea el mejor para todos ellos. Existen algoritmos que destacan en simular aspectos específicos del transporte lumínico y un buen motor de renderizado puede aprovechar ese hecho para combinar algoritmos de forma inteligente. Por ejemplo, versiones del algoritmo de radiosity o instant radiosity pueden usarse en una primera fase para computar un mapa de luz difusa de la escena. En una segunda fase se puede utilizar alguna de las variantes de path tracing para calcular la luz especular y la luz refractada, haciendo consultas al mapa de luz difusa cuando sea necesario. Finalmente una ejecución de photon mapping puede usarse para calcular las causticas con un mayor nivel de detalle.
Una buena explicación de este uso combinado de algoritmos se puede encontrar en \cite{Hery2013}.

\medskip

Por ultimo y aunque no era el objetivo de este proyecto, una mejora interesante seria implementar una interfaz de usuario que permitiese configurar la escena de forma fácil y cómoda. Por ejemplo una interfaz con Qt con una vista OpenGL, que permita previsualizar la escena, cargar modelos tridimensionales y colocar las luces y la cámara facilitaría mucho el hecho de configurar escenas y crear nuevos renderizados.


\section{Perspectivas de futuro}

Teniendo en cuenta los resultados obtenido, que la GPU utilizada no es de las del mejores del mercado y que nuestra implementación es muy optimizable. Considerando además la velocidad a la que avanza la tecnología de las tarjetas gráficas y que ya existen aproximaciones a soluciones de iluminación global con tiempos de renderizado interactivos, creemos que en pocos años sera habitual disponer de aplicaciones que hagan uso de este tipo de algoritmos en tiempo real. 