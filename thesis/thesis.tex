\documentclass[12pt, a4paper,twoside]{tesi_upf}

\usepackage[utf8]{inputenc}
\usepackage[T1]{fontenc}
\usepackage[spanish]{babel}
\usepackage[cam,a4,center,frame]{crop}
\usepackage{graphicx}
\usepackage{times}
\usepackage{url}
%\usepackage{garamond}
\pagestyle{plain}

\usepackage{makeidx}
\makeindex

%BIBLIOGRAPHY STYLE
\bibliographystyle{vancouver}


%LANGUAGE
\selectlanguage{spanish}

\addto\captionsspanish
  {\renewcommand{\contentsname}{\Large \sffamily Índice}}


%YOUR INFORMATION HERE:

\title{Trazado de rayos estocástico en la GPU}
\subtitle{}
\author{Joaquim Romo}
\thyear{2014}
\department{GTI}
\supervisor{Javier Agenjo}

\includeonly{
	content/1_Introduccion,
	content/2_Fundamentos,
	content/3-ResearchObjectives,
	content/4-ResearchPlanAndMethodology,
	content/5-SummaryOfPriorWork,
	content/6-Significance
}

\begin{document}


\frontmatter

\maketitle

\cleardoublepage




\selectlanguage{spanish}
\section*{\Large \sffamily Resumen}

El presente trabajo tiene como objetivo el estudio y la implementación de un algoritmo de renderizado con raytracing estocástico en la unidad de procesamiento de gráficos (GPU).
Se ha elegido realizar la implementación en una arquitectura de este tipo debido a las ventajas que ofrece en cuanto a tiempo de ejecución, gracias a la gran capacidad de cómputo en paralelo que ofrecen las arquitecturas de GPU actuales.


La primera parte del trabajo se dedica al estudio teórico del algoritmo de pathtracing, se comentan algunos conceptos físicos básicos relacionados con el transporte de luz y su interacción con los materiales así como las ecuaciones matemáticas y las técnicas estadísticas necesarias para la comprensión y correcta implementación del algoritmo.


En una segunda parte se discuten las tecnologías involucradas, concretamente el uso que se hace de la arquitectura CUDA, la librería OptiX y su funcionamiento y la implementación del algoritmo que se ha realizado sobre estas.  

%DIFFERENT TABLES OF CONTENT:

\tableofcontents
\listoffigures

\addcontentsline{toc}{chapter}{Index of figures}
\listoftables

\addcontentsline{toc}{chapter}{Index of tables}

\cleardoublepage

%HERE STARTS THE TEXT:
\mainmatter
\chapter{Introducción}

\section{Contexto}

En el entorno de la imagen generada por computador siempre ha sido un reto tratar de generar imágenes lo más realistas posibles. Para ello un gran número de investigadores se han dedicado a diseñar algoritmos que simulan o imitan el comportamiento y la interacción de la luz con los materiales. Estos algoritmos que tratan de simular de forma realista el comportamiento de la luz son generalmente conocidos como algoritmos de iluminación global.

\medskip

Estos algoritmos, por lo general, suelen tener una complejidad computacional muy elevada y el tiempo de cómputo necesario para obtener un resultado satisfactorio en escenas complejas era un factor limitador en su aplicación práctica. Por ello las aplicaciones que hacen uso de gráficos 3D en tiempo real típicamente se centran en la iluminación local o directa de los objetos de la escena y simulan la iluminación indirecta mediante técnicas que aun sin tener un fundamento físico ofrecen una mayor credibilidad para el ojo humano. Estas técnicas suelen ser algoritmos de postprocesado que se aplican en espacio de pantalla, por ejemplo \emph{ambient occlusion} o \emph{directional occlusion}. 

\medskip

Sin embargo en los últimos años se han realizado grandes avances en las arquitecturas de las unidades de procesamiento de gráficos (GPUs), en especial la gran capacidad de cómputo en paralelo debido al elevado número de microprocesadores que forman estos dispositivos. Con tal de aprovechar estos avances en el hardware los fabricantes de GPU han desarrollado librerías de computo genérico (OpenCL, CUDA), que ofrecen gran libertad al programador para implementar sus propios algoritmos.

\medskip

Estas mejoras han permitido realizar implementaciones de algoritmos de iluminación global en las GPUs que son mucho más rápidos que las implementaciones típicas en la CPU, permitiendo reducir el tiempo de cómputo de varias horas o días a minutos e incluso a tiempos interactivos dependiendo de la GPU y algoritmos utilizados.

\clearpage

\section{Algoritmos de iluminación global}

Se conoce como algoritmos de iluminación global aquellos que tratan de simular distintos aspectos del comportamiento de la luz en su interacción con los objetos de una escena tridimensional. Algunos de ellos están pensados y optimizados para fenómenos concretos, mientras que otros tratan de recrear fielmente todos los aspectos del transporte de luz.

\medskip
En esta sección revisaremos por encima algunos de los algoritmos clásicos. Téngase en cuenta que no es el objetivo de este trabajo hacer un análisis exhaustivo de todos los algoritmos ni dar una explicación detallada de cada uno de ellos. Si el lector desea mas información sobre alguno de ellos, se han citado las fuentes originales a las que puede remitirse.

\subsection{Ray tracing}

Aunque no se trata de un algoritmo de iluminación global propiamente dicho, el algoritmo de ray tracing original, desarrollado primeramente por Appel (1968)\nocite{Appel1968} y posteriormente ampliado en \cite{Whitted1979}, es relevante por la influencia que ha tenido en el campo de los gráficos generados por computador y por que ha servido de base para métodos de iluminación global desarrollados posteriormente. 

\medskip

En este algoritmo se lanzan rayos desde la cámara a través de los pixels de la pantalla y se comprueba si interseccionan con los objetos de la escena. En la versión de Whitted además se lanzan rayos recursivamente si el objeto es reflectante o refractante y hacia las luces, para comprobar si al objeto le llega luz directamente.

 
\subsection{Radiosity}

Radiosity fue el primero de los algoritmos de iluminación global que se desarrollaron. Inicialmente el algoritmo fue desarrollado en los años 1950 para aplicarlo al problema de la transferencia de calor. En 1984 fue modificado y adaptado por \nocite{Goral1984} Cindy M. Goral, Kenneth E. Torrance, Donald P. Greenberg y Bennett Battaile, investigadores de la universidad de Cornell para su aplicación en la generación de imagen sintética.

\medskip

Este algoritmo trata de resolver el problema de la iluminación indirecta entre superficies puramente difusas sin tomar en cuenta la reflectancia especular.

\medskip

El funcionamiento del algoritmo, en lineas generales, se basa en dividir la escena en pequeñas unidades de área, llamadas parches, que deberían funcionar como diferenciales de área. Luego a través de una serie de iteraciones se intenta balancear el flujo de luz emitido, reflejado y absorbido entre todos estos parches.

\clearpage

\subsection{Path tracing}

El algoritmo de path tracing \cite{Kajiya1986} sea posiblemente el primer algoritmo capaz de solucionar completamente la ecuación de renderizado.

\medskip

Este algoritmo empieza como el ray tracing clásico, lanzando rayos desde la cámara hacia la escena, pero cuando un rayo intersecciona con un objeto se lanza un rayo en una dirección aleatoria para tener una estimación de cuanta luz indirecta llega a ese punto. Este rayo aleatorio, a su vez es evaluado recursivamente siguiendo el mismo procedimiento. 

\medskip 

Evidentemente este proceso de trazar un rayo desde la cámara y hacerlo rebotar por la escena para obtener un estimación de la luz es muy impreciso, por lo que es necesario repetir el proceso varias veces, y hacer la media entre los resultados obtenidos para obtener una solución satisfactoria. 

\subsection{Bidirectional path tracing}

El algoritmo de Bidirectional path tracing \cite{Lafortune1993} fue desarrollado como una extensión al algoritmo de path tracing de Kajiya. En esta modalidad los rayos primarios no solo se lanzan desde la cámara, sino también desde las fuentes de luz. Estos caminos de luz, se calculan del mismo modo que los de la cámara. Se guardan los puntos de intersección de los caminos de la cámara y los de la luz y en una ultima fase se unen estos dos grupos de puntos para obtener la evaluación final del camino.

\medskip
La principal mejora de este algoritmo respecto a su antecesor, es que es capaz de funcionar mejor y converger mas rápido hacia una solución correcta en escenas complejas, en las que las fuentes de luz no son fácilmente visibles desde la mayoría de puntos de la escena.

\subsection{Metropolis light transport}

Siguiendo en la linea de los dos algoritmos anteriores, otra notable mejora llego con el llamado Metropolis light transport \cite{Veach1997}. Este algoritmo parte de la base del path tracing bidireccional, pero en vez de confiar en crear muchos paths hasta converger a una solución aceptable, utiliza un método conocido como algoritmo de Metropolis-Hastings para generar varias mutaciones del mismo path. 

\medskip
Este algoritmo desata todo su potencial cuando se trata de renderizar interacciones complejas entre materiales, que normalmente serian muy costosas de renderizar con los algoritmos que hemos comentado anteriormente. Por ejemplo cáusticas, intereflexiones especulares-difusas, etc.

\clearpage

\subsection{Photon mapping}

Todos los algoritmos que estamos viendo tratan la luz como partículas y no como ondas, pero este algoritmo lo hace de un modo aun mas explicito.
El algoritmo de Photon mapping \cite{Jensen1996} empieza lanzando rayos (fotones) desde las fuentes de luz. Cuando estos fotones interseccionan con un objeto de la escena se decide aleatoriamente y según las propiedades (BRDF) del material si el fotón sera absorbido, dispersado especularmente o dispersado difusamente. Las posiciones finales donde los fotones son absorbidos se guardan en un mapa (kd-tree) de fotones para la siguiente fase.

\medskip
La siguiente fase, llamada final gathering, realiza un ray tracing de la escena y en cada intersección consulta el mapa de fotones para ver la cantidad de luz que llega a ese punto.

\medskip
Este algoritmo sobresale entre todos los demás cuando se trata de renderizar cáusticas, pero en cambio, puede producir errores cuando se renderizan superficies difusas si el numero de fotones no es muy grande.


\subsection{Instant radiosity}

Este algoritmo, desarrollado por Keller en 1997 \nocite{Keller1997}, combina las ideas de los algoritmos de radiosity y photon mapping. Igual que el radiosity original, este algoritmo, en principio, solo funciona para superficies puramente difusas.

\medskip
La idea general consiste en lanzar fotones desde las fuentes de luz (como en photon mapping) e ir guardando sus posiciones. La diferencia principal radica en que en la fase de renderizado, estos fotones son tratados como luces puntuales (VPLs, del inglés Virtual Point Lights) con orientación, es decir que tienen un vector normal. La ventaja es que una vez generados estos VPL es posible renderizar la escena mediante una API gráfica tradicional acelerada por hardware, como OpenGL o DirectX, o con ray tracing.

\medskip

Además del algoritmo de instant radiosity básico existen modificaciones del mismo que combinan las ideas de los algoritmos de path tracing con el de instant radisioty. En este conjunto de algoritmos encontramos el instant radiosity bidireccional y el metropolis instant radiosity.  


\clearpage

\section{Tecnologías involucradas}

El propósito de esta sección es explicar las principales tecnologías utilizadas durante el desarrollo de este proyecto. Debido al alcance de este trabajo la mayoría de tecnologías que comentaremos en este apartado son tecnologías especificas de las GPU.

\subsection{Arquitectura de las GPU modernas}

Las arquitecturas de GPU modernas siguen una paradigma conocido como SIMD (del inglés, Single Instruction Multiple Data), que consiste en la capacidad de un procesador de ejecutar la misma instrucción en paralelo sobre datos distintos. Es decir, que se ejecuta el mismo proceso en varias unidades de computo, pero los datos que trata cada unidad pueden ser distintos.

\subsection{Shaders}

Los fabricantes de GPUs empezaron a explotar esta capacidad de computo en paralelo, ofreciendo a los programadores de gráficos la posibilidad de programar ciertos puntos del proceso de renderizado efectuado por las GPU, con los llamados shaders. Los shaders son pequeños programas que se ejecutan en la GPU y sirven para programar funcionalidades. Tradicionalmente existían dos tipos de shaders: los vertex shaders, que se ejecutaban para cada vértice de cada primitiva a pintar, y los pixel o fragment shaders que se ejecutaban para cada pixel rasterizado.

\medskip
Según el paradigma SIMD, solo puede ejecutarse un único shader en cada GPU pero los datos pueden ser distintos: en el caso de los vertex shaders, por ejemplo, cada unidad de procesamiento tendrá acceso a las coordenadas geométricas de un vértice, las coordenadas de textura de ese vértice, etc. Es decir que un vertex shader ejecutara exactamente las mismas operaciones para cada vértice en paralelo. 

\clearpage

\subsection{Capacidad de computo genérico}
La principal limitación de los shaders es que solo trabajan con datos relacionados con el renderizado de gráficos (coordenadas de vértices, coordenadas de textura, vectores normales, texturas, etc). Si un programador quería utilizar la capacidad de computo en paralelo de las GPU para problemas distintos al renderizado por rasterizado de gráficos, debía buscar la manera de codificar los datos del problema en forma de vértices y texturas, lo cual podía resultar tedioso o complicado.

\medskip
Debido a esta necesidad los fabricantes de GPU empezaron a buscar una forma de ampliar las capacidades de computo que ofrecían sus dispositivos y desarrollaron plataformas de programación genérica en GPUs. La primera de estas plataformas fue la desarrollada por Nvidia, con el nombre CUDA, para sus tarjetas gráficas. Posteriormente se desarrollo OpenCL, un estándar de computo en paralelo tanto para GPUs, como para CPUs, soportado por la mayoría de fabricantes de hardware.

\subsection{CUDA}

CUDA es la plataforma de computo genérico sobre GPU desarrollada por Nvidia para GPUs de Nvidia. El compilador nvcc (NVidia C Compiler) permite tanto compilar programas con una parte en el host (CPU) y otra parte en el device (la GPU), como compilar kernels, que se ejecutaran en la GPU.

\subsection{OptiX}

OptiX es una librería de ray tracing sobre CUDA desarrolada por Nvidia. Optix esta diseñado con la idea de ofrecer un framework para ray tracing lo más genérico y programable posible, tratando de no limitar las posibilidades del programador.

\clearpage

\section{Objetivos}

El principal objetivo de este trabajo es realizar una implementación de un algoritmo de iluminación global, lo más completo posible, usando tecnologías de GPU para aprovechar las ventajas que ofrecen estos dispositivos en cuanto a tiempo de ejecución.

\medskip

Para ello, un primer objetivo necesario era estudiar los distintos algoritmos y decidir cual se adaptaba mejor al objetivo de este trabajo, tomando en cuenta la completitud del mismo, es decir, cuan realistas son los resultados ofrecidos por el algoritmo. Otro factor importante en la decisión del algoritmo a utilizar es que sea paralelizable, factor indispensable para poder realizar una implementación eficiente en la GPU.

\medskip

Una vez decidido el algoritmo, otro objetivo parcial necesario para alcanzar el objetivo primario es realizar un estudio teórico detallado del algoritmo en cuestión, y de las técnicas y conceptos relacionados con el mismo, con tal de tener una comprensión de este lo bastante profunda como para realizar una implementación seria.

\medskip

Finalmente sera necesario plasmar los conocimientos teóricos adquiridos en una implementación de un renderizador con iluminación global. 






\chapter{Fundamentos teóricos}

\section{Unidades Radiométricas}
Se conoce como radiometría al estudio de las radiaciones electromagneticas. Ya que la luz visible es una onda electromagnetica los algoritmos de renderizado que buscan el realismo se fundamentan sobre conceptos radiométricos. Por ello en esta sección haremos una pequeña introducción sobre algunos conceptos básicos que nos permitiran entender mejor los algoritmos de iluminación global.  
\subsection{Flujo}

El flujo radiometrico mide la cantidad de energia radiante por unidad de tiempo. Sus unidades son Watts o Joules/segundo.

\begin{equation}
\Phi = \frac{dQ(t)}{dt}
\end{equation}

\subsection{Irradiancia}
La irradiancia representa el flujo incidente en una superficie y se mide como el flujo radiante por unidad de area y sus unidades son de $W/m^2$ 

\begin{equation}
E = \frac{d\Phi}{dA}
\end{equation}

\clearpage

\subsection{Angulo solido}
El angulo solido no es una unidad radiométrica en si mismo pero es un concepto geométrico necesario para poder explicar otros conceptos radiométricos además de otros apartados del presente trabajo.

Podemos entender el concepto de angulo solido como la extension del angulo a las tres dimensiones.
\begin{figure}[h]
\centering
\includegraphics[width=2in]{Solid_Angle.png}
\caption{Definición de angulo solido \cite{Haade}}
\end{figure}

El angulo solido se mide como el área proyectada sobre una esfera de radio unitario. Sus unidades son adimensionales y son llamadas stereorradianes $[sr]$.
\begin{equation}
\Omega = \frac{A}{r^2}
\end{equation}


Usando coordenadas esféricas $\Theta = (\phi , \theta )$ podemos definir el angulo solido diferencial como:

\begin{equation}
d \omega _ \Theta = \sin \theta d \theta d \phi
\end{equation}

Informalmente resulta sencillo entender el angulo solido si pensamos en \emph{cuan grande se ve un objeto}. Supongamos una superficie perpendicular a la dirección de visión del observador: si este objeto esta muy cerca, diremos que subtiende un angulo solido mayor que la misma superficie a una mayor distancia en la misma dirección.

\clearpage

\subsection{Radiancia}

La radiancia, también llamada intensidad por algunos autores \cite{Kajiya1986, Immel1986}, es probablemente la unidad radiometrica mas importante en lo que concierne al presente trabajo.

Esta unidad mide la irradiancia por unidad de angulo solido.

\begin{equation}
L = \frac{dE}{d\omega} = \frac{d^2\Phi}{d\omega dA\cos \theta} 
\end{equation}


\clearpage

\section{BRDF}

La función de distribución de reflectancia bidireccional (de ahora en adelante BRDF, por sus siglas en inglés), definida por primera vez por \cite{Nicodemus1965} Nicodemus (1965), es un función que define la respuesta a la luz de una superficie opaca, tomando como parámetros dos vectores unitarios que definen las direcciones de entrada y salida de la luz. Más formalmente, la BRDF mide la relación entre la radiancia diferencial reflejada en la dirección de salida y la irradiancia diferencial entrante en el ángulo sólido diferencial alrededor del vector de entrada

\begin{equation}
f(x, l, v)=\frac{dL(x \to v)}{dE(x \gets l)} 
\end{equation}

donde $l$ es el vector unitario que apunta en la dirección opuesta a la de entrada de la luz y $v$ es el vector unitario que apunta en la dirección de salida de la luz.

La BRDF solo esta definida para vectores $l$ y $v$ tales que $n \cdot v > 0, n \cdot l > 0$, siendo $n$ la normal de la superficie.

\begin{figure}[h]
\centering
\includegraphics[scale=0.5]{Rendering_eq.png}
\caption{BRDF $l = \omega_i, v = \omega_o$ }
\end{figure}


Para obtener la radiancia total reflejada en un punto $x$ en la dirección saliente $v$ es necesario integrar sobre el angulo solido en el dominio de la hemiesfera centrada en $x$.

\begin{equation}
\label{eq:radiance_integral}
L _ o = \int_{\Omega_x} f(x, l, v) L_i(l) (l \cdot n) d\omega_i 
\end{equation}

\clearpage

\subsection{Propiedades de la BRDF}

Una BRDF debe cumplir ciertas propiedades para que sea físicamente plausible.
En primer lugar debe cumplir la ley de conservación de la energía. En el caso que nos ocupa esto significa que una superficie puede absorber luz, transformándola en calor, o puede reflejarla pero en ningún caso puede reflejar mas energía lumínica que la que recibe.

\begin{equation}
\forall l, \int_{\Omega_x} f(x,l,v) (n \cdot v) d\omega_o \leq 1
\end{equation}

En términos informales esta ecuación significa que la integral de toda la luz reflejada debido a un rayo de luz entrante nunca podrá ser superior a la luz entrante por ese rayo.

\medskip
Además también debe cumplir el \emph{principio de reciprocidad de Helmholtz}, esto significa que si intercambiamos los vectores $l$ y $v$ su valor se mantiene. Este hecho cobra sentido si pensamos que la BRDF es una característica intrínseca de cada material y que al intercambiar los vectores $v$ y $l$ el angulo entre ellos sigue siendo el mismo.

\begin{equation}
f(x, l, v) = f(x, v, l)
\end{equation} 


\clearpage

\section{Ecuación de renderizado}

La ecuación de renderizado fue desarrollada en los años 80 simultaneamente y de forma independiente por distintos autores \cite{Kajiya1986, Immel1986}. Se trata de una ecuación integral que unifica y formaliza los distintos algoritmos de renderizado, ya que hasta ese momento no existía un marco de trabajo teórico común.

\medskip
Existen varias versiones de esta ecuación, según el autor que la use, que en general se pueden clasificar en dos tipos: las que integran sobre la hemiesfera, que se corresponde con la ecuación propuesta por Immel y las que integran sobre la unión de las superficies de la escena, que es la version propuesta por Kajiya.
\medskip

Consideremos la ecuación \ref{eq:radiance_integral}  y consideremos que además de dispersar luz una superficie también puede emitir luz, siendo $L_e$ la radiancia de la luz emitida, entonces tenemos la ecuación de renderizado.

\begin{equation}
L _ o = L_e + \int_{\Omega_x} f(x, l, v) L_i(l) (l \cdot n) d\omega_i 
\end{equation}

Es decir, que la radiancia total $L_o$ que sale de un punto $x$ es igual a la suma de la radiancia emitida por ese punto en la dirección de salida $v$ mas la integral de toda la radiancia que llega a ese punto y es reflejada en la dirección de salida.
\medskip

Lo significativo de esta ecuación es que resulta muy intuitivo derivar algoritmos de renderizado de la misma: se evalúa para cada punto a pintar y se evalúa $L_i$ recursivamente hasta que se cumpla determinada condición.
\medskip

El problema es que no parece factible encontrar una solución analítica de esta ecuación y por este motivo se aplican métodos de integración numérica para aproximar una solución.

\clearpage

\section{El método de montecarlo}

El método de montecarlo se trata de un método de integración numérico para integrales definidas sobre un dominio de dimension arbitraria, del tipo:
\begin{equation}
I = \int_D f(x)dx , D \subseteq \mathbb{R}^m
\end{equation}

Sabemos que la esperanza de una función continua se define como la integral de la función por la probabilidad de $x$. Y que podemos estimar la esperanza calculando la media de los valores que toma la función en puntos aleatorios escogidos independientemente y con la misma distribución.

\begin{equation}
E(f(x)) = \int f(x)p(x)dx \approx \frac{1}{N} \sum _{i=1} ^N f(x_i) 
\end{equation}


El método de montecarlo se basa en este hecho para estimar el valor de una integral definida tomando muestras aleatorias sobre el dominio $x_1, x_2, ..., x_n \in D$ y aplicando:

\begin{equation}
\label{eq:montecarlo}
I = \int_D f(x)dx \approx \frac{1}{N} \sum _{i=1} ^N \frac{f(x_i)}{p(x_i)} 
\end{equation}

Siendo $p(x_i)$ la probabilidad de tomar una muestra $x_i$ concreta de entre todas las posibles en el dominio $D$. En el caso de tomar las muestras sobre una distribución uniforme en $D$:

\begin{equation}
\forall x_i, p(x_i) = \frac{1}{\int _D dx}
\end{equation}

\begin{equation}
I \approx \frac{\int _D dx}{N} \sum_{i=1} ^N f(x_i)
\end{equation}

El error en una estimación de este tipo se reduce a medida que $N$ crece.
\clearpage

\subsection{Muestreo de importancia}
    
Otra forma de reducir el error a parte de tomar mas muestras es tomarlas de forma mas inteligente. Anteriormente hemos supuesto que tomamos las muestras de una distribución uniforme sobre el dominio pero el método de montecarlo no impone ninguna limitación en este aspecto. Lo que implica que podemos tomar las muestras de otro tipo de distribuciones que sean mas apropiadas para cada caso. Por ejemplo tomando mas muestras en aquellas partes del dominio de integración que sean mas interesantes o importantes para nuestros propósitos.

\medskip

Para ello basta con tomar las muestras $x_1, x_2, ..., x_n$ según la distribución usada y substituir $p(x_i)$ en la ecuación \ref{eq:montecarlo} por la probabilidad correspondiente.

\subsection{Muestreo estratificado}

El muestreo estratificado es otro metodo para reducir la variancia de la estimación. En este caso lo que se hace es dividir el dominio de la integral en regiones y aplicar montecarlo para cada región.  




\chapter{Research objectives}

UbiCALab.


\section{A Section}

UbiCALab.

\begin{table}[h]
  \centering
  \begin{tabular}{|l|l|}
   \hline
    0 & 0 \\ \hline
    0 & 0 \\ \hline    
  \end{tabular}
  \caption{Prova de taula}

\end{table}

\subsection{A Subsection}

UbiCALab.

\section{Another Section}

UbiCALab.. 

Incorpora una figura amb el logo de la UPF (Figura \ref{fig:logo}) {perque surti a l'index de figures. 
\begin{figure}[h]
  \centering
  \includegraphics[scale=.5]{logo_upf.png}
    \caption{Exemple de grafic}
    \label{fig:logo}
\end{figure}

\chapter{Research plan and methodology}

Lorem ipsum dicam tation maiorum vel ea, an nam minimum pertinax vituperatoribus, mea no illud tibique nostrum. Odio aperiri dolorem in sit, utinam labore adipiscing cu vel. Nam porro choro cu, nam virtute maluisset voluptatum in, per te vero iuvaret praesent. Eos ut volumus pertinacia interesset, tota debitis abhorreant mei et, clita electram pri at. Ea fabulas admodum theophrastus sea, id nostro omnium periculis sed. Vim id nulla aliquando. Id mel habeo nostrud mandamus, laudem legimus fastidii quo ea, ut iriure utamur quo.

Cu sit nobis omnesque scripserit, nec albucius perfecto salutatus ex. Novum debet propriae vis cu, sed ea suscipit accusata, nam minim salutandi voluptatum ut. Quaerendum definitiones est ut, no quo postea prompta scaevola. Nec at choro soleat causae, per ut docendi denique, discere veritus no sed. Ei sed sale delicatissimi mediocritatem, eligendi deterruisset ut mei, ignota offendit ocurreret at mel. Numquam blandit gubergren an pro, mei cu quot dolore recteque. Suscipit aliquyam principes eam ut, id error dicit perpetua qui.

Eam mentitum noluisse instructior cu, ad scripta nonummy aliquyam eos. Augue possit gloriatur mel an, mea et dolorum suavitate assueverit. At ius dolores persecuti moderatius, id qui justo tantas dissentiet. Integre invenire imperdiet ut mea, harum alienum officiis te nam, integre qualisque conceptam in vis. Viderer consetetur vim id, sumo tempor complectitur vim an. His te aeterno tamquam integre, nec agam aliquando repudiandae id, te fabellas verterem sea.

Per dicit veniam delicata ut, dicunt corpora mei eu. His no alienum accusata accommodare, eu nam clita audire invidunt. Mutat debitis mediocrem duo te, ex vis prima labore inermis. Usu eu dictas possim vocent. Suas meliore ancillae ad eum, eam ei cibo viderer similique.

Usu ne apeirian deserunt expetendis, sed in iisque abhorreant definiebas, quo nulla inermis appareat cu. Ei nam possim docendi pertinax, ut duis veritus detracto pri. Animal percipit ius ne, ne iisque quaeque philosophia pro, ut sed nobis nullam adolescens. Persius principes omittantur per an. Mel quot nostrud labores at, ad vis latine persius petentium. Cum partiendo neglegentur mediocritatem ut, per simul aliquando tincidunt no, falli nominavi molestiae ex nam.

Saperet inciderint in vim, vis quod civibus accusata in. Et ullum facilisi accommodare nec, decore utroque definitionem an vel, id duo kasd facer latine. Ea sumo ullamcorper per, ex pro nobis efficiendi concludaturque, vix eu verear luptatum ocurreret. Ocurreret intellegat te eum, ius ea cibo epicurei dignissim. Diam elit quaeque cum ut. Est dicat summo prodesset te, no vel harum saepe sensibus, et ignota altera doming qui.

Sit altera regione feugiat in, vim ad augue mediocritatem, duo reque choro theophrastus ea. Ipsum apeirian voluptatum duo no, nam ut ornatus accumsan percipitur, est no posse suscipiantur vituperatoribus. Errem legendos partiendo est cu, est clita signiferumque cu. Ea nulla scripta signiferumque eam, sea vidit dolores officiis ei, nibh saepe in nam. Ad vim justo dolor reprimique, eum ad labitur dissentias scriptorem, cu tempor mentitum assentior eum. Id vis mucius percipit invidunt.

Vel at ullum intellegam constituam, mea dicta affert perpetua ea. Eripuit blandit iracundia nec ea, tale timeam molestie cum te. Cum ei kasd mazim, per ei delenit ullamcorper, ad sale vocibus temporibus usu. An nonummy oporteat phaedrum sed, eu unum deserunt incorrupte mei. His an possim delicata, ei accumsan constituam mea. Ei viris constituam pro, cu inimicus argumentum duo.

Cu regione veritus persequeris vis, tation nusquam percipit in pri, vis no labitur deseruisse. Ius in reque fierent. Quo et erant recusabo honestatis. Iusto accusam liberavisse ex mei, ut cum nobis primis accommodare, viderer vocibus assueverit ut nec. Id odio oporteat instructior pri. Nam sint illud vituperatoribus no, zzril ubique dissentias id nec.

Cu est legimus reformidans. Nam cibo mucius dolores ad, prima posse similique vix te. Mea at nemore mandamus assueverit, sit viris option scaevola cu. At omnis decore nostro pro, aliquyam explicari salutatus ei mea, sea no falli incorrupte. Graeco bonorum lobortis ut sed, ut pro discere legimus, an sea volumus consetetur.

\chapter{Summary of prior work (if any)}

\include{content/6-Significance}



%%%%%%%%%%%%%%%%%%%%%%%%%%%%%%%%%%%%%%%%%%%%%%%%%%%%%%%%%%%%%%%%
\cleardoublepage
%\phantomsection
\addcontentsline{toc}{chapter}{Bibliography}
\bibliography{bib/library}



\backmatter
\printindex





\end{document}


%DO NOT CHANGE THIS PART OF THE DOCUMENT
\usepackage{fancyhdr}
\pagestyle{fancy}
\fancyfoot{}
\fancyfoot[RO]{\thepage}
\fancyfoot[LE]{\thepage}



\usepackage{multind}
\makeindex{authors}

\index{authors}{Einstein}

\printindex{authors}{Author index}
